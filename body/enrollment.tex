\chapter{UNIBO入学指南}              

大学各学院在年末会公布来年招生公告,公告中标明了每个专业将招收的新生名额,一般分为欧盟学生名额,非欧盟学生名额,及针对中国学生的 Marco Polo 计划名额,值得注意的是,非Marco Polo计划的中国学生将竞争非欧盟学生的名额(国际生名额)。 

\section{入学注册}

\subsection{Almawelcome 网站注册入学专业考试}
新生必须使用自己的税号(尚未拥有税号的可使用自己的个人身份信息)在大学新生网站 Almawelcome 上注册,得到自己的用户名及密 码。使用此用户名及密码登录后,可在已激活的入学考试名单中选择自己将报考的专业,打印申请表及缴费单。交完入学考试的费用后请妥善保管好银行的收据,这是注册入学考试的凭证。新生须按照各专业公告中的时间到指定的地点参加入学考试, 考试规则各有约定,考试时带上本人身份证件及缴费单收据。各专业的公告中一般还会通告专业考试放榜的时间,新生可留心关注。请注意各专业申请的截止日期。 如果考试顺利,但申请截止日期错过的话,依然会被取消入学资格。

\subsection{参加意大利语考试或持有意大利语证书 (免考)}
正式注册大学的另一必要条件是新生的意大利文水平(如果意大利语授课)。未持有大学认可的意大利文证书的新生必须在九月初,专业考试之前,参加一项由各学院举办的意大利文水平测试,通过后方可注册。若持有大学认可的意大利文证书,则只需在前往学院秘书处正式注册时(既通过专业考试后)提供证书的复印件即可。(备注:计划生有B1 证书免考,国际上则需要持有 B2 证书,否则需要加试意大利语测试) 




\subsection{无专业入学考试的情况}
如果新生选报的专业本学年未设置入学考试,新生只需在 Almawelcome 网上正式注册,开放之后办理注册入学(immatricolazione)手续即可。值得注意的是,部分专业可能出现有入学考试,但申请名额未满,考试不必进行的情况,此时新生仍需 按照有入学考试的情况报名申请缴费,然后等待公布考试成绩的日期之后办理正式入学手续。 
\subsection{正式注册入学}
从指定日期起,新生(如有入学考试则需是通过入学考试的新生)可开始办理大学正式注册入学手续。新生先在 Almawelcome 网站上填写并打印入学申请表格,打印第一期学费的缴费单(每学年可选分期支付或一次付清)后到银行缴费,最后到所属秘书处交材料完成所有入学手续并领取学生手册及学生卡。 

所需材料:
\begin{itemize}
\item 填写好并签名的入学申请表格 
\item 缴纳第一期学费的银行收据 
\item 意大利文水平证书(如果未参加大学组织的意大利文水平测试) 
\item 经翻译双认证的高中毕业证书或毕业证和学位证(从大学毕业时取回)
\item 个人寸照
\item 申请居留的回执
\item 护照复印件 
\end{itemize}




\section{入学之后}
大学网站系统发达,各类信息应有尽有。Bologna 大学网站:www.unibo.it 。关于个人信息管理,可以登录, studenti.unibo.it 之后可以, 预约考试(AlmaEsami),交学费,开注册考试证明(Autocertificazione, 当办居留的材料, 申请语言学习(意大利语,英语) 。。。
如果想要快速找到自己专业的网站,推荐通过谷歌搜索,输入“专业名称 unibo”即可。在自己专业的网站上能查到详细的课程表(Orari di lezioni),学习计划(Piano di studio),在学习计划里面能找到细致的课程描述,参考书,考试方式介绍等等。

\subsection{ALMAWIFI 无线网}
ALMAWIFI 是博洛尼亚大学提供的无线网,覆盖所有校区和学生宿舍,注册生都可以登录。登陆无线网需要大学发的用户名及密码,账户格式 ming.xing@studio.unibo.it . 无线网的下载速度非常快,可能1秒好几mb,但是不支持P2P种子下载。连接大学无线网的方法: 选择大学无线网Almawifi,用户名和密码分别为登陆大学账户的邮箱及密码。

\subsection{次年注册手续}
入学一年以后,直至取得新的学位之前,学生都必须无条件的按年缴付学校规定的学费。学费缴付之后,即成功注册了新一学年的学习。除此之外,外国学生还需要注意,旧的居留证过期后,应随时向秘书处递交更新之后的居留证复印件,或者新的居留证申请回执复印件,因为若缺少有效的居留证件,新一年的注册将不会被系统识别,这将会影响到日后考试的注册以及分数登记(verbalizzazione)。 居留条的递交能保持你的学籍3个月有效。

\subsection{学费}
学费单可从 studenti.unibo.it 上,登录之后打印。通过上面的支付码可在意大利境内任何一家 UniCredit 银行缴付学费。同时,也可利用信用卡进行网上学费缴纳。学费须在截止日期前缴付完毕,若逾期未交,将处以几十欧的罚款。 

\subsection{学分制系统介绍}
本科课程学制一般为三年制,总共需要修满180 学分(cfu - crediti formativi universitari),每年60学分的工作量。研究生课程学制一般为两年制,共120学分。研究生入学要求本科课程和研究生的相匹配,如果缺少一些必修学分,往往会收到面试评估,面试过的话才能读研,否则得重新读本科。 

\subsection{选课及学习计划(Piani di studio)}
每一个专业均为学生安排了必修课程和自选课程 (attività formative opzionali) ,学生可以按各自的兴趣,在给定的范畴内选择自己喜欢的课程、实习或研讨活动等(insegnamenti, tirocini, laboratori, seminari, ecc. ) 。选课要求所有的注册生在学校规定日期内,向系秘书处 (Segreteria della facoltà) 递交自己的选课结果,除此之外,大学网站上也开通了有关学习计划的相关网页,学生们可以直接在网上进行填写,具体操作如下: 登陆 https://studenti.unibo.it/ 然后选择 login, 输入自己大学的 username 和 password 之后, 便可以进入 studenti online 页面,选择 piano di studio 即可开始选课。一般在选课截止之前,学习计划的相关网页都会允许学生自由登陆, 并对之前的选择进行修改。一旦选课截止,该网页便会关闭。之后所有关于选课的问题,均须到秘书处咨询解决。每一学年都有选课的机会,但每一学年的注册生,只可选择或修改本年度或往年的学习计划,例如,刚注册的新生只可递交第一学年的学习计划,不能也无权选择第二学年的学习计划,但第二年的注册生,除了可为本年度课程制定学习计划外,还可对第一年已选择的课程进行修改。需要注意的是,学习计划的生效,只针对大学的正式注册生,即已按时缴付学费的学生, 否则,即便递交了学习计划,也会被认为无效。此外,若某年度课程中有待选项目,则学生必须递交过学习计划后,该年度的所有课程对应的考试才会生效,也就是说,若考试通过,可被系统识别,否则系统检测不到这些课程便会显示错误(包括该年度 内所有的必修课,若不递交学习计划,系统也同样不能识别)。学习计划的填写和递交日期,每年都不尽相同,具体详情请查询学院相关网页以便获得准确信息。 

\subsection{考试制度}
考试采用 30 分制,即 30 分为满分,及格分18 分,有时教授为表扬考试出色的学生,除了给他们打 30 分,后面还要加上“Lode”,以示不同。值得注意的是,以自然科学系为例,换成美国的ABC档的话,30L属于A档,29-30属于B档,26-29属于C档,所以你会发现身边意大利人的考试分数接近30的特别多。
考试之前,学生须提前5天在网上预约 (prenotazione) 。考试结束后,预约的界面分数登上了,才算彻底的结束。一般情况下,教授都会为学生统一安排时间进行登分(verbalizzazione)。
所有的专业都会按照学校的要求,在一个学年内,为每门课程安排大约5次考试机会(5 appelli) 。因此,在课程结束之后的一年时间里,若对某次考试成绩不满意,尽可注册下一次的考试,直至考取满意的分数,再请教授把成绩登入系统和记分册。需要注意的是,许多教授在给学生登记成绩时,有特殊的规定:对那些参加过若干次考试的学生,只允许这些学生登记最近一次通过考试的成绩,并不择优登记,也就是说,即使前几次考试的成绩都高于最近的这一次考试成绩,也只能登记最近的这次成绩(当然,前提是成绩在 18 分以上,包括 18 分), 因此,在决定参加下一次考试之前,最好权衡之后再行动。 只有已经注册的学生(即已缴纳学费并在秘书处留下有效居留证复印件)及已将待考科目纳入学习计划的学生,方可参加该门考试,否则,即使考试通过也无法对成绩进行登记。 
有时,教授会特别针对跟班上课和在家自学的同学,准备不同的试卷,因此在注册考试时,需特别留意。对于分阶段上课的同一门课程,最后只登记一个综合成绩。

\subsection{毕业论文/设计}

毕业时间:每学年都有 3 次毕业机会,每个学院毕业时间,毕业机会次数都可能不一样。需要单独查询,以下为大体研究生毕业时间。
\begin{itemize}
\item 第一阶段:7 月 12 日~20 日
\item 第二阶段:10 月 17 日~22 日 11 月 15 日~30 日 12 月 10 日~17 日
\item 第三阶段:3 月 15 日~31 日
提交毕业申请时还需交付毕业证书的定制费 (Pergamena) 
\end{itemize}

\noindent 提交毕业申请的截止日期(对应上述 3 个阶段): 
\begin{itemize}
\item 第一阶段:5 月 15 日
\item 第二阶段:9 月 15 日
\item 第三阶段:1 月 15 日 
\end{itemize}
迟到提交会收到罚金。

\subsection{实习 (Tirocinio)}
学校会不定期在网站上张贴布告(Bando) ,向学生提供各种有关实习机会的消息,学生可根据自身兴趣选择参加,一般需递交个人简历,或进行面试。 实习与毕业论文,在与相关导师协调后,也常能结合在一起完成。实习的选择范围很宽泛。除了学校推荐的公司和企业外,学生还可自行寻找实习对象,但该实习对象须与校方签订相应合同 (Convenzione),实习方可生效。此外,在校生与毕业后不满 18 个月的毕业生均可参加这些实习项目。对于在校生,实习往往是必须的,因此学校方 面对实习的时间也有相应规定,每工作 25 小时即可 
赚取一学分,例如,若要获得 10 学分,则一共需要工作至少250 小时。而对于毕业生,学校方面则没有任何时间限制。 
\subsection{留学回国人员证明}
在毕业后如何到罗马的中国驻意大利大使馆教育处开具此证明 
前提准备: 需要去当地学联网站(Bologna 大学中国学联的网站是:www.boxue.it)或学联办公室领取并填写《赴意大利留学人员登记表》。

\section{学费减免及奖学金}

大区学习权利机构 ER-GO 为学生提供各项便利, 包括学费减免、奖学金、住宿、食堂及国际交换协助等。ER-GO 每年七月公布各项目的公告及申请截止日期, 请关注网站 www.er-go.it 。 
学费减免分多等,最多减免一半,只有在有奖学金的情况下,学费才全免。头一年申请凭借家庭状况排名,从第二年往后,根据学分排名,同样学分时,根据平均分排名。奖学金按租房情况分 In sede 、Pendolare 和 Fuori sede 三大类,金额递增。In sede,为黑户,pendolare,房东开免费居住证明,fuori sede,有正规房租合同的付费证明。In sede收到的奖学金是Fuori sede的一半。所以一定注意在截止日期之前提交合同复印件。对于房产在中国,父母工作在中国的大部分留学生来说,都是 fuori sede 的选项。

\subsection{申请基本流程}
\begin{itemize}
 \item 网上填写申请表格 \\https://www.er-go.it
 \item 点击 login, 输入大学 username 和 password
 \item 进入主页面, 点击选项: "per compilare/visualizzare il modulo per I benefici a concorso a.a.
2016/2017"
 \item 在截止日期前确认表格中所填写内容并提交。
 \item 打印申请表格签字 
 \item 将申请表附上各项证明材料在截止日期前寄出 
 \item 等候 ER-GO 公布结果:往往11月初左右,如果材料有问题,可以补充。12月底开始发第一批奖学金。
 \item 8月10号之前达到要求学分,会收到第二批奖学金。
\end{itemize}
注:如果是需要申请学生宿舍的同学,请
务必将所有认证材料在 8 月 10 日之前办好并寄给 ER.GO, ER.GO 需要提前审核材料并且 在两个星期之后公布第一批入住宿舍资格名单,如果又缺少材料或者材料有问题的同 学,ER.GO 将会给你发补全材料的信件,如果在规定日期到期之前将材料补全并且通过 审核,则可以在第二批名单(最终名单)中获得入住宿舍的资格。

\subsection{申请奖学金的基本学分要求}

\subsubsection{学分要求}

\subsubsection{本科}
\begin{tabularx}{\textwidth}{ |X|X| }
  \hline
  年份 & 要求学分\\
  \hline 
  第一年  & 25  \\
  第二年  & 80  \\
  第三年  & 135  \\
  \hline
\end{tabularx}



\subsubsection{研究生}
\begin{tabularx}{\textwidth}{ |X|X| }
  \hline
  年份 & 要求学分\\
  \hline 
  第一年  & 30  \\
  第二年  & 80  \\
  \hline
\end{tabularx}


\subsubsection{借学分(Bonus)}
因为特殊情况,没有达到指定要求学分的同学也可通过借学分的方式获得奖学金。可借学分如下表。一共只能借一次,借剩下的学分可以来年接着用。\\
\begin{tabularx}{\textwidth}{ |X|X| }
  \hline
  年份 & 可借学分\\
  \hline 
  第一年  & 5  \\
  第二年  & 12  \\
  第二年  & 15  \\
  \hline
\end{tabularx}

\subsection{家庭经济状况证明需准备之材料}
\begin{itemize}
 \item 家庭成员亲属关系证明(成员、关系、人口,父母离异亦须证明)
 \item 各家庭成员上一年的年度收入(无业务收入等都须开出相应证明) 
 \item 家庭拥有的房产证明(如无房产则提供租房合同,分期付款未清证明等)
\end{itemize} 

\subsection{办理奖学金的注意事项}
以上各项材料均需公证翻译成意大利文并经双认证。过去 Er.Go 开始办理奖学金的时间是从7月中旬开始,每年略有变动,详情请关注 ERGO.it 或者Bologna大学中国学联官方论坛公告(www.boxue.it/bbs ) 