
% \thispagestyle{empty}                   %elimina il numero della pagina
% \topmargin=6.5cm                        %imposta il margina superiore a 6.5cm
% \raggedleft                             %incolonna la scrittura a destra
% \large                                  %aumenta la grandezza del carattere
%                                         %   a 14pt
% \em                                     %emfatizza (corsivo) il carattere
% Dedico questa tesi alla mia famiglia e \\ tutti coloro che hanno creduto in me.                    %\ldots lascia tre puntini
% \newpage                                %va in una pagina nuova
% %
% %%%%%%%%%%%%%%%%%%%%%%%%%%%%%%%%%%%%%%%%
% 


\chapter{博洛尼亚欢迎你}                 %crea l'introduzione (un capitolo

\section{学联致辞}

各位新同学:
             欢迎来到博洛尼亚!欢迎加入博洛尼亚大学学生学者联谊会的大家庭!
          博洛尼亚大学是意大利国际化程度最高的大学之一, 也是最早出现中国留学生身影的地方, 随着中意两国政府教育合作的不断深化, 从2004年开始博大不断增加在中国的招生规模, 至今已经有超过 1000名中国学生在博洛尼亚大学学习.
          ASSCUBO(Associazione di Studenti e Studiosi Cinesi dell’Università di Bologna)是博洛尼亚大学中国学生学者联谊会的意大利文缩写, 是第一个在意大利注册的地方学联, 也是第一个被博洛尼亚大学承认的外国学生协会, 接受大学的经费支持.2009年10月, 博洛尼亚学联举办了第一次全意大利规模的中国留学生活动: 首届全意中国留学生篮球联赛; 在汶川, 阿圭拉和玉树地震灾害后, 学联分别举行了三次赈灾募捐活动, 总共募捐8819.38欧元;  2010年5月, ASSCUBO代表中国学生第一次参加博洛尼亚大学学生会的选举, 产生了意大利大学史上第一名中国留学生议员; 学联也多次接受意大利媒体的访问, 为增强意大利社会对中国的了解与认识起到了积极的作用.
          博洛尼亚学联以服务留学生为己任, 为了更好地做到这一点, 学联在近年建立了网上论坛, 筹办了图书室, 和商户合作为学生提供 更多便利; 作为一个留学生展现自我和锻炼才能的舞台, 所有加入学联的干事都是义务为大家服务的,促进了博大中国留学生友爱互助的和谐氛围的形成,并成为优良传统继承至今; 为了增强海外的华人的凝聚力,服务更多莘 莘学子,学联开展形式多样的文体活动,如意语角,篮球赛等。每逢中国传统佳节 学联都会举办活动邀请一些当地华人 意大利朋友参加,借此增强与华人的沟通与合作,积极弘扬中国文化,更好地融入当地社会,学联的执 委会由主席、副主席、秘书处、学习部、体育部、文艺部、外联部、新生部和宣传部组成,除此之外我们还有博士生联谊会,记者站 摄影协会 民乐团 篮球队 合唱团以及话剧社等新兴社团。学联每年各个部门、各个社团都将招收新的 干事与成员,欢迎大家的踊跃参与。 



\section{中国学院简介}


         中国学院协会成立于2005年10月,她是博洛尼亚大学自1088年创建以来各学院精心培育外国学生的传统精神的新体现。

          中国学院协会有9个创始会员:除了博洛尼亚大学之外,还有艾米里亚·罗马涅地区政府、博洛尼亚省政府、博洛尼亚市政府、博洛尼亚商会、地区商会联盟、博洛尼亚展览会、博洛尼亚工业协会、博洛尼亚中小企业协会、博洛尼亚手工业全国联合会、Alma Mater基金会。它们代表了艾米里亚·罗马涅地区最重要的行政、文化、经济、工业及社会机构。\\\\
          \noindent 中国学院院长:Prof. Roberto Grandi\\
          指导教师:刘怡\\
		   E-mail: tutor@collegiodicina.it, \\
		   Tel: 051/2098608, Cell: 346/7999580 \\
		   秘书处:Tel: 051/2099767\\
		   E-mail: segreteria@collegiodicina.it\\
			地点:Palazzo Paleotti (3楼),Via Zamboni, 25,40126,Bologna\\
官方网站:http://www.collegiocina.it\\
Facebook: www.facebook.com/collegiodicina\\
新浪微博: www.weibo.com/collegiodicina

\section{博洛尼亚大学孔子学院}

意大利博洛尼亚大学孔子学院由中国人民大学与博洛尼亚大学于2009年3月2日共建成立。外方院长为博洛尼亚大学法学院Marina Tiameo教授,中方院长为中国人民大学文学院徐玉敏副教授。 博洛尼亚大学前校长Ivano Dionigi教授与中国人民大学校长陈雨露教授,现为孔子学院总部理事会理事。博洛尼亚大学孔子学院下设一所孔子课堂和若干个汉语教学点,可设有汉语初级课程、中级课程、高级课程和中国文化体验课程。博洛尼亚大学孔子学院近年来连续举办了一系列精品中国文化推广活动,如牡丹亭巡演、中意高端论坛、“武林汉韵”大春晚、中国文化节等,在当地社会引起了强烈反响。\\\\
 \noindent 开放时间:\\
周一:15.00-17.00\\
周二到周五:10.00-13.00, 15-17.00.\\
地址:Palazzo Paleotti (3楼) , Via Zamboni 25,40126 Bologna \\
邮箱:istituitoconfucio@unibo.it \\
意方代表: Marina Timoteo 电话:+39-3382988772 传真:+39-0512098540\\
邮箱:marina.timoteo@unibo.it \\
中方代表:徐颖\\
邮箱:ying.xu@unibo.it  \\
官方网站:http://www.istitutoconfucio.unibo.it

\section{博洛尼亚大学简史}
\begin{itemize}
\item 1088年大学成立: 是第一所从教会独立,有自由教育权利的大学。此外,Irnerio法案\cite{irnerio}标志着西方大学的诞生。
\item 12世纪学术自由: 腓特烈一世国王承认大学是自由的教师与学生组成的社会团体,并且给予游学以保护,这是第一次,学术的自由被认可。
\item 13世纪学术中心: 在这个时期,博洛尼亚大学逐渐变为国际化的学术中心,来自欧洲各地,2000多名学生在博洛尼亚学习。大学为自己的独立自治和任何外界势力对抗。
\item 14-15知识的地平线: 从14世纪开始,在法律学院的加入之后,博洛尼亚大学变成了医学,哲学,数学,天文学,逻辑学,修辞学,语法学,法学的学术中心,大学并且变成当时知识分子必须去的地方。
\item 16-18从博洛尼亚到世界, 从世界到博洛尼亚: 医学,哲学,自然科学,数学,工程还有经济学是博洛尼亚大学的主要学科,博洛尼亚在这些领域的研究变成了当时世界该领域的权威。从博洛尼亚到世界: 从世界到博洛尼亚:博洛尼亚往外发送,并吸引世界各地的学者和科学家,缔造宽阔的国际关系网。
\item 1888大学之母: 为庆祝大学800年生日,全球大学代表会聚于此,给予博洛尼亚大学大学之母的荣誉。在博洛尼亚大学的庆祝仪式演变成了国际教育节。
\item 20世纪新挑战: 大学在世界各大研究中心和单位中依然是无可争议的核心地位,她在一个活跃和复杂的全球框架中,继续和世界上的各大研究单位合作。
\item 1988年签署国际协议: 庆祝大学900年生日时,来自全欧洲500所大学的校长相聚博洛尼亚大学,共同签署了一份协议,声明大学在当代社会中的关键和自主的角色。
\item 1999年博洛尼亚进程: 为统一全欧洲高等教育的博洛尼亚进程而签字。
\item 2012年大学展望未来: 这个年度,博洛尼亚大学采用了新章程,结束了挑战未来教育,科研和国际化的改革。
\item 2015-2016年世界大学排名
	\begin{itemize}
		\item QS\cite{qs}: 全球第204位,意大利境内第2位
		\item TIMES\cite{times}: 全球第203位, 意大利境内第4位
		\item USNews\cite{usnews}: 全球第142位,意大利境内第1位
		\item 上海交大世界大学排名\cite{shanghai}: 全球第259位,意大利境内第7位
		\item 维基百科大学排名\cite{weiji}: 全球第26位,意大利境内第1位
	\end{itemize}
\end{itemize}

\section{博洛尼亚之最}
\begin{itemize}
\item 西方世界最古老的大学
\item 世界最长的长廊40km
\item 世界上第一位创立阳历日历的 Papa Gregorio XIII 的故乡。
\item 世界整容手术的诞生地
\item 意大利第一批中国华侨聚集的地方
\end{itemize}

