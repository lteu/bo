\chapter{博洛尼亚中国学联}              

\section{学联简介}

\subsection{学联性质}
中文名:博洛尼亚大学中国学生学者联谊会 
意大利语名:Associazione di Studenti e Studiosi Cinesi dell'Università di Bologna (ASSCUBO)
简称:博大学联
性质:博大学联是在博洛尼亚学习的中国留学生和学者的联谊团体, 既是全意学联下属的学生组织, 也是在意大利内政部注册的合法协会, 是非政治性非盈利性的群众团体。 
办公室地址: Via Saffi 28, Bologna

\subsection{学联宗旨}
立足于服务广大留学生和学者, 帮助解决实际困难. 
加强EMILIA-ROMAGNA大区中国学生、学者间的联系和互动。 
通过举办与参与各种形式的活动, 达到丰富留学生的课余生活的目的, 为中国学生接触和融入意大利 社会起到促进作用。 
 传播中国文化,促进中意文化交流。 

\subsection{学联会员}
博洛尼亚学联的主体是在博洛尼亚大学就读的中国留学生, 近年来随着中意两国之间的教育合作不断深化,来博洛尼亚大学读书的中国留学生也越来越多, 目前已超过 700 名学生,而且近两年规模也在逐渐扩大, 每年会有接近 200 名新生注册 Emilia- Romagna 大区各所大学, 其中选择加入学联的同学成为学联会员。除博洛尼亚本部之外,我们的成员还包括 Rimini, Cesena, Forli, Ravenna 分校区的 同学. 

\subsection{学联构成}
博大学联所有成员分为志愿工作者与一般会员. 学联工作的执委会由主席, 副主席,理事,秘书处, 学生部 (新生部),宣传部, 学习部, 外联部,文艺部和体育 部组成; 我们还有博士生联谊会, 龙之道运动社团, 记者站, 摄影协会, 民乐团, 篮球队, 合唱团, 以及话剧社等新兴社团. 学联每年各个部门, 各个社团都将招收新的干事与成员. 

\subsection{部门职}
\begin{itemize}
\item 学联主席:拥有学联重大事务的最终决策权,从宏观上统领学联事务
\item 副主席:协助主席工作。 
\item 理事: 人员管理决定, 财务决定和监督, 活动执行, 决议落实, 学联规定把关。理事由学联工作时间久或贡献重大的同学担任。
\item 秘书处:配合各部门的工作;负责学生资料的统计与管理;第一时间发放活动通知。
\item 学生部(新生部):负责对《新生手册》的修订工作;并每年派遣同学赴各地语言学校为新来同学提 供实用信息;开展龙之道运动社团的相关活动。
\item 宣传部:对每次活动进行宣传工作,制作活动海报,论坛管理。
\item 学习部:综合学习资料; 举办学习经验交流会; 开展图书角和意大利语角; 促进学业间交流。
\item 外联部:负责对外联系,是学联与外界沟通的窗口; 负责活动赞助和拓展合作关系。
\item 文艺部:举办文艺演出,才艺比赛,为有文艺特长的同学们提供发挥特长的空间。
\item 体育部:通过组织体育活动和比赛,丰富同学们的课余生活,以体育活动带动中意文化交流。 
\end{itemize}

\subsection{学联图书}
Biblioteca del centro di Studi e Informazioni Amilcar Cabral. Via San Mamolo, 24 Bologna


\subsection{合作伙伴}
学联在发展的过程中离不开外界的支持与帮助, 经过 多年工作的开展, 我们已和博洛尼亚大学中国学院, 博洛尼亚孔子学院,意中基金会,欧洲华人报,博 洛尼亚商会等建立了合作关系。 

\section{学联联系方式}
\begin{itemize}
\item 网站 boxue.it \\
注意: 在网站上可以绑定自己的邮箱,近而收听学联活动通知。
\item 微博 weibo.com/asscubo
\item QQ群 58292281
\item 论坛 http://www.boxue.it/bbs
\end{itemize}

\section{加入学联的方式}
想加入学联的同学可以来学联办公室Via Saffi 28进行办理,需要先发邮件到seg.csser@gmail.com预约,或加秘书长微信 602723872 预约,一周办理一次,具体时间预约通知。
办理时请务必携带 
\begin{itemize}
\item 护照复印件
\item 学生卡
\item 如果办理学联卡,费用为10欧。
\end{itemize}