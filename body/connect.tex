
% \thispagestyle{empty}                   %elimina il numero della pagina
% \topmargin=6.5cm                        %imposta il margina superiore a 6.5cm
% \raggedleft                             %incolonna la scrittura a destra
% \large                                  %aumenta la grandezza del carattere
%                                         %   a 14pt
% \em                                     %emfatizza (corsivo) il carattere
% Dedico questa tesi alla mia famiglia e \\ tutti coloro che hanno creduto in me.                    %\ldots lascia tre puntini
% \newpage                                %va in una pagina nuova
% %
% %%%%%%%%%%%%%%%%%%%%%%%%%%%%%%%%%%%%%%%%
% 


\chapter{交通通信}                 %crea l'introduzione (un capitolo

\section{城市公告交通 (Collegamenti Urbani)}

\subsection{车票的种类}
车票分为次票,十次票(City Pass),天票(Biglietto Giornaliero),月票(Mensile)和年票(Abbonamento Annuale)。除了年票要到公交公司指定销售点办理外,其他票都可以在烟草店(Tabacchi)买到。
\begin{itemize}
\item  次票:打票后75分钟有效,期间可以任意乘车。车上买1.5 欧,烟草店买1.3 欧
\item  十次票 12欧
\item  天票 5 欧
\item  月票 27 欧 (27 岁以下),36 欧 (标准票,36欧的周末可以带一个朋友坐车)
\item  年票 220 欧(27岁以下),300 欧(标准票)办理需要携带照片和证件,比如护照或身份证,当场可以领取。
\item  机场大巴 6 欧
\end{itemize}
乘坐公交车,在中途转车到乘坐另一辆车,次票需要再打印一次, 在打票凭证的前面会有 C 字样,表示换乘。

\subsubsection{罚款}
\begin{itemize}
\item  65 欧,如果5天内交罚款
\item  87 欧 ,(标准罚款)
\item  260 欧,如果迟迟不交
\item  6 欧,如果乘车没带年票。
\end{itemize}
如果被罚款,如果不交罚款,查票人会问你询问护照,居留,税号等个人信息,建议最好当场补交罚款,否则会记录个人诚信档案,对以 后在意大利等欧盟等国家办理信用卡,找工作会遇到一些不必要的麻烦。为了不遇到这种情况,大家最好上车打票,不要存在侥幸心理。\\

罢工时,站牌会有公布相关信息,上面能看到当天或者提醒本星期的罢工时段。或者下载诸 如”moovit” app 即可实时提醒罢工时间,因集会游行而造成的线路改动,也能查询到实时的公交车时间。\\
TPER 销售点:1)Autostazione 里面 2)Via Marconi 4, ang. via Lame



\section{火车 (Treni)}

\subsection{意大利火车系统简介}

官方网站: trenitalia.it 该网站可以查询火车时刻表,罢工信息,购买火车票等。
意大利的火车按速度和行驶范围一般可以分为多个档次:
\begin{itemize}
\item  Regionale ,大区火车,站多,相当于国内火车的普快。火车票灵活,类似公交票,座位和乘坐车次可以灵活选取,上车前必须打票,未来得及打票,上车后需要及时寻找列车员进行说明,他们会写上时间。
\item  Intercity(IC) | Intercity Notte , 带有隔间的火车,站点相对少,相当于国内火车的快车。每个车票对应固定车次,固定座位。
\item  Freccia Bianca 特快,车票对应固定车次,固定座位。
\item  Freccia Rossa,Freccia Argento 动车,车票对应固定车次,固定座位。周六有2人行的折扣。
\item  Italo 动车,非国有企业,但性价比更高,电子车票无需打印,只拍个电子票号码即可。车票对应固定车次,固定座位。动车车票提前买折扣大。
\end{itemize}
注意:\\
乘车:到达火车站乘车,请观看大厅的公告板(Tabella luminosa)的出发(Partenza)一栏,不要被 Arrivi 的时刻迷惑。在公告板上可以得知要坐的火车是哪个站台,是否晚点(Ritardo) \\
关于车票:只有Regionale的火车票可以像公交票一样灵活使用,未打票的火车票有效期往往3个月。动车的火车票如果购买的是经济舱,是无法更改,退换的。\\
关于Italo特色车厢:Italo动车的cinema车厢可以看意大利电影,但是需要自配耳机。意大利动车的速度往往能达到 300km/小时。\\
关于慢车:如果所去地方仅是途经站,须看站台附近的打印版固定时刻表,黄色为出发,白色为到达。因为公告板(Tabella luminosa)主要显示的是起始站和终点站,途径车站滚动显示,不太方便查询。\\
此外,下载app trenitalia 或 pronto treno 快速购买想要的火车票。以及在trenitalia官网上注册cartafreccia
young, 如果是 26 岁以下通常可以享受 freccia 的青年优惠价, 并且会时常有邮件推送 freccia 的折扣信息。

\section{长途大巴 (Pullman)}
在意大利以及欧洲短途旅行或者长途旅行时,除了火车和飞机以外,还可以选择廉价大巴出行,一般来说目前有很多国外网站和 app 在做,比较著名的有 flixbus。通常去同一个城市, 坐大巴花的费用是不到火车费用一半的价钱。缺点是大巴班次一天只有三班,不像火车班次 很多,不过如果时间不着急可以选择乘坐大巴。如果是出国或者在意大利境内比较长途的旅行,还可以乘坐夜间大巴,通常是 10 个小时左右到达目的地,晚上出发,早上就到了, 不耽误白天的时间,而且如果打折,通常费用就1欧搞定。缺点是可能在车上睡得不是很舒服。

\section{出租车 (Taxi)}
Bologna市出租车运营公司为Contabo,从火车站打车到飞机场往往15欧左右。\\
网站 http://www.cotabo.it 。\\
打车电话为 051 372727 \\
同时大家可以下载APP 'Taxiclick' 来预约出租车\\
关于 Cotabo 的一段简介,"Co. Ta. Bo., (Cooperativa Tassisti Bolognesi) 是ER大区最大的出租车司机联合会,包括706个司机,其中539在博洛尼亚地区,服务全年24小时。营业范围覆盖全博洛尼亚,打表计价。公司提供手机app - Taxiclick,在AppStore,PlayStore都可以下载到,功能简洁,全面,当今市面应用很广。本款App还提供注册,网上支付,月底账单汇总等功能。Cotabo可根据客户需求,量身定做多种服务。


\section{私人租车 (Noleggio auto)}
驾龄超过一年之后才有租车的资格。租车可以在网上申请,然后到租车办事处领车。根据车型和淡旺季,租金40-100多欧/天不等。\\
欧洲流行的租车公司网站有(费用由低到高):
\begin{itemize}
\item https://www.europcar.it
\item http://www.avisautonoleggio.it/default.aspx 
\item https://www.hertz.it/rentacar/reservation/
\end{itemize}


\section{航空 (Aerei)}

\subsection{廉价航空}
在欧洲旅行首推乘坐各家廉价航空公司的航班,如果购票及时很有可能买到超级便宜的机票,而且航线广,欧洲各大城市都有涉及。以下为几家常用链接航空网站:
\begin{itemize}
\item www.ryanair.com
\item www.easyjet.com
\item www.vueling.com
\item www.opodo.it
\end{itemize}

\subsection{关于回国}
打算回国时可能遇到的情况
\begin{itemize}
\item 第一种情况:第一次申请居留,手里只有居留条,不可以回国,除非签证在有效期内。
\item 第二种情况:持有过期居留,同时有居留条。可以乘坐国航直飞,不能在欧盟内其他国家转机,可以在俄罗斯,迪拜等转机。
\item 第三种情况:持有有效居留卡,怎么飞都是允许的,包括在英国,俄罗斯,迪拜,土耳其转机。
\end{itemize}

\section{邮政|通信|网络}

\subsection{邮局常用到的几个业务}
\begin{itemize}
\item 邮政储蓄。
\item 邮寄业务。挂号信(Raccomandata)是法律性质强的邮寄方法。尤其办理奖学金材料时用到。邮寄前先窗口领取表格填写,避免延误时间。领取时可以不排队,但是要观察工作人员心情,挑看起来脾气好的人,那边领取。
\item 居留申请。邮寄居留表格之前,不要忘记购买16欧的印花税,还有保险。
\item 其他杂务: 水电煤气费用, 保险, 医药费用等, 每张缴费单子(bolletta)需多缴纳1,3 欧的手续费(commissione).

\end{itemize}

\subsection{快递}
意大利境内主要快递公司包括:HL, TNT, UPS, FEDEX, POSTEITALIANE 等等。
\begin{itemize}
\item DHL:服务更好,送货速度更快,但价格更贵。从国内邮寄文件,小物品比较合适,一般3-5个工作日。
\item TNT:意大利最常用的快递公司,价格更便宜,国内EMS包裹到意大利一般也由TNT接管。
\item FEDEX: 是美国联邦快递,是快递中寄往中国海关税收得最低的公司。
\item POSTEITALIANE: 是意大利邮政,优点是便宜,缺点是时间太慢,不论是在意大利境内的 投递还是邮寄回中国都很慢,少则一个月,多则三个月不等。并且在寄回国以后还需要
缴纳一定程度的海关税。
\item 人肉快递:价格往往10-15欧一公斤,国内邮寄费用另算。可在学联QQ群内寻找。
\end{itemize}

\subsection{通信}
固定电话意大利的国际电话区号是 0039 (+39)。意大利各地区区号都以 0 开头,如米兰 02,罗马 06,博洛尼亚 051 等,所以从中国拨打意大利博洛尼亚市的固定电话 的方式为:0039+051+固定电话号码。意大利手机使用的手机网络模式与中国一致,均为 GSM 模式,所以大部分的中国手机在意大利均可以直接使用。意大利主要的手机运营商有四个 wind、 tim、vodafone 以及 tre。 

\subsubsection{Wind}
www.wind.it:意大利最流行的运营商之一。所属号段:320、327、328、329、380、 388、389 等。
主要业务:
\begin{itemize}
\item All Digital: 10欧元/月,2GB 无限网络,500分钟,无限短信
\item All inclusive YOUNG:  12 欧元/月,3GB 无限网络, 300分钟,3000条短信。
\item All inclusive:  12 欧元/月,1GB流量,500 分钟,500 条短信
\end{itemize}
呼叫中国: CALL YOUR COUNTRY ,第一分钟 0.18 欧元为接通费,每分钟 0.02 欧元打中国,每4周1欧

\subsubsection{Tim}
www.tim.it:意大利国家电信。所属号段:331、333、334、335、336、 337、338、339 等。
主要业务:
\begin{itemize}
\item Tim Young Junior:6欧/月,500MB网,60分钟电话,拨给TIM号免费,60条短信。
\item Tim Young:9欧/4周,3GB的4G网络,1000条短信,不包电话。
\item Time Young Full:14欧/4周,3BG的4G网络,1000条短信,500分钟电话。
\end{itemize}
呼叫中国:Tim International new, 开通费9欧,每分钟0.01欧元,无月租。

\subsubsection{Vodafone}
www.vodafone.it:总部英国的电信公司。所属号段: 340、346、347、348、349 等。 
主要业务:
\begin{itemize}
\item Vodafone Young:  9欧/4周,500MB网,100分钟电话,50条短信。
\item Vodafone Under 30:  12欧/4周,2GB的4G网,200分钟电话,200条短信。
\item Relax:34欧/月,3GB的4G网(全欧洲,美国,加拿大),无限电话和短信。
\end{itemize}
呼叫中国:MyCountry4G,12欧/4周,2GB4G网,接通费19欧分,打中国,打意大利都是 0.01欧/分钟。



\subsubsection{Tre}
www.tre.it:李嘉诚投资的企业。所属号段:392、393 等。 天天电讯:所属号段:377。 
主要业务:
\begin{itemize}
\item  Super Internet:5欧/30天,100MB/天,即3GB/月,不过,超过流量会扣钱。
\item  Super Internet 8GB:10欧/30天,2GB/一周,8GB/月。
\item  All-in: 10欧/月,2GB的4G网络,400分钟电话,400分钟短信。
\end{itemize}


手机充值方式一般有三种:
\begin{itemize}
\item 去 Tabacchi 烟店、Bar、中国商店购买充值卡。 
\item 去各大电信公司的官方网站在线充值,优点是可以使用中国银行的visa 信用卡。 
\item 下载电话公司APP,用APP充值,同时也能知道自己的流量剩余,套餐使用情况。
\end{itemize}
打电话回国,中国的国际电话区号为 0086,因此从意大利打往中国	的电话的拨号方式为: 国内固定电话:0086+国内区号(去掉前面的 0) +固定电话号码。例如打到天津:0086+22+固定电话号码。 国内手机:0086+国内手机号码(不需要在手机号码前加0)。

 其他注意事项:意大利所有手机都是可以保号转网的,即你可以变换电话公司而不用担心更换号码带来的麻烦。而且意大利全境无漫游费,所有在佩鲁贾或者锡耶纳办理的电话卡在意大利的其他城市也是一样的资费。 
\subsection{网络}

\begin{itemize}
\item 宽带上网:意大利主要网络运营商包括 fastweb、alice(tim 公司品牌)、infostrada(wind 公司品牌)及vodafone 等。办理需要带着证件,税卡,银行账号,家庭住址。
\item 户外上网:麦当劳,政府的市政厅或者大学,机场往往都提供免费网络。
\item 无限卡上网:需要购买一种类似 U 盘的 key 上网,优点是上网灵 活,插在电脑上比较便携。
\end{itemize}
